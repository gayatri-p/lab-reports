\section{Discussion \& Conclusion}
Our experiments utilizing the four probe method allowed us to accurately determine the resistivities of various samples, including both semiconductors and metals.
Using the the variation in voltage with the applied current, we have estimated the resistivity values of the three given materials to be

\begin{itemize}
    \item Aluminium, $\rho = (3.131 \pm 0.184) \times 10^{-5}\,\,\Omega$ cm
    \item n-Germanium, $\rho = (19.322 \pm 0.387)\,\,\Omega$ cm
    \item n-Silicon, $\rho = (21.549 \times 0.431)\,\,\Omega$ cm\\
\end{itemize}

These are close to the actual values provided by the manufacturer for n-Si ($24 \pm 1\,\,\Omega$ cm) and n-Ge ($18 \pm 1\,\,\Omega$ cm). However, there is a significant deviation in the case of Al ($2.8 \times 10^{-6}\,\,\Omega$ cm for pure Al), which could be due to the fact that we are using commercial-grade Al foil.\\

Additionally, We observed that the resistivity of semiconductors decreased with an increase in temperature, inline with theoretical predictions.
We were able to calculate the band gap of
a semiconductor sample through these temperature-dependent measurements, which came out to be 
\vspace{-1em}
\begin{align*}
    E_g = (0.626 \pm 0.002) \text{ eV}
\end{align*}

This is close to the literature value of 0.680 eV. There is a round $-8$\% deviation, which could be attributed to the fact that the band-gap varies with temperature and the literature value is defined at 303 K (30$^\circ$C).

As the temperature increases, more electrons jump from the valence band to the conductance band. These electrons can now conduct the current and the conductivity increases.
And hence the resistivity decreases.\\

We also saw that the temperature-dependence of resistivity of Aluminium metal. We saw that there was an increase in the resistivity as temperature increases, which is caused due to increased phonon scattering. We approximated a linear order fit over the temperature range used in the experiment, to estimate the temperature coefficient of Al, which came out to be $\alpha = (8.475 \pm 0.442) \times 10^{-3}$ K$^{-1}$. This is roughly of the order of $\alpha$ defined for pure Aluminium, $4.308\times 10^{-3}$ K$^{-1}$. There was also a large error associated with the measurements (see Fig. 10), primarily due to the error associated with the measurement of $W$ ($\sim 7$\%).\\

Note that fluctuations in supply voltage, carrier injection, or impurities in the sample material could have lead to errors which were propagated to our final results.
We also made assumptions about the uniformity of resistivity in our samples, which may not always hold true.

\section{Precautions and Sources of Error}

    \begin{enumerate}
        % \item The sample should be of uniform thickness.
        \item Instability in the data due to improper contact could lead to errors, so the springs should be tightened carefully.
        % \item The formula for $\rho$ is valid for semi-infinite /very large surface in comparision with the probe distance, which could be a source of error in the calculations.
        \item Al used in the foil is commercial grade, while standard resistance is for pure Al.
        \item Variation of doping or impurities in the sample could also contribute to errors.
        \item Sufficient time should be given for the temperature to stabilize before noting the measurements.
    \end{enumerate}
