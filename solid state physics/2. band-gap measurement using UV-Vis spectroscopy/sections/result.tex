\section{Discussion \& Conclusion}

In this experiment, we were able to determine the bandgaps for three different direct band-gap materials. These are:

\begin{itemize}
    \item ZnTe: $(1.859 \pm 0.026)$ eV
    \item CdS: $(2.074 \pm 0.013)$ eV
    \item ZnO: $(2.539 \pm 0.359)$ eV\\
\end{itemize}

We have found that the bandgap values obtained from this method are slightly lower than the literature values for all three materials (2.26 eV, 2.42 eV and 3.35 eV for ZnTe, CdS and ZnO respectively). Hence the percentage errors are 18\%, 14\% and 24\% respectively. This could be attributed to impurities contaminating the sample. Since we did not use a vacuum chamber, there could also be dust and other particles interfering with the light matter interaction.

We can also see that the error in the band gap of ZnO is significantly higher than the other two materials. This is expected as the absorbption vs. wavelength plot obtained for this material was very noisy (Fig. \ref{zno}). This could be because that particular sample of ZnO deposition might have worn off or contaminated heavily. Uneven sample thickness could also contribute to the inaccuracy in the measurement.

\section{Precautions}

    \begin{enumerate}
        \item Make sure to not hold the sample by hand to avoid contamination.
        \item Wait for a few minutes after switching on the light source for it to provide maximum intensity.
        \item Take proper dark and reference readings to ensure proper calibration.
    \end{enumerate}

\section{Applications}

In conclusion, UV-Vis spectroscopy is an important analytical tool used in many fields, including chemistry, biology, environmental science, and pharmaceuticals. It can be used to identify and characterize molecules, measure their concentration, and determine the purity of a sample. Indirect bandgap materials also provide a more continuos and wider range of absorbance compared to direct bandgap materials which makes them better suited for use in solar cells.