\section{Discussion \& Conclusion}

In this experiment, we looked at how the dielectric constant and the capacitance of BaTiO$_3$, vary with frequency. We saw that the dielectric constant lowers as the total polarisation decreases and the relaxation times of various polarisation mechanics result in slower alignment of polarisation vectors in the direction of the electric field as compared to the frequency of the electric field. We also compared it to the variation in standard MLCC and DCC to observe a general trend.
	
We also investigated BaTiO$_3$'s phase transition. The dielectric constant of BaTiO$_3$ is seen to increase up to the Curie temperature and then decrease symmetrically after.
From this we have obtained the value of Curie temperature as
\begin{align*}
    T_C = (135 \pm 5)\,\,^\circ\text{C}
\end{align*}
This shows that barium titanate exhibits ferroelectric behaviour below the Curie temperature (where it exists in a tetragonal phase with a net dipole moment) and paraelectric behaviour above $T_C$ (where it exists in a cubic phase without a net dipole moment). We may infer that the Curie temperature is unaffected by the frequency of the applied electric field because it is the same for each frequency curve.

We can probably attribute the slight deviation in $T_C$ to the impurities in the sample since the measured curie temperature deviates from the predicted value of around 120 $^\circ$C.

We found that frequency had little effect on the variable resistance. Only the amplitude of the signal on the oscilloscope was impacted by adjusting the variable capacitor's capacitance. But since the sample's resistance is not necessary, we may infer that the balancing resistance is essentially independent of C.
	
By plotting $\log(\frac{1}{\epsilon}-\frac{1}{\epsilon_C})$ vs. $\log(T-T_C)$, we calculated the diffuseness parameter of the sample at different frequencies,

\begin{itemize}
    \item at 10 kHz: $(1.40 \pm 0.05)$
    \item at 25 kHz: $(1.35 \pm 0.03)$
    \item at 35 kHz: $(1.57 \pm 0.12)$
    \item at 50 kHz: $(1.46 \pm 0.09)$\\
\end{itemize}

Here we observed no general trend and all the values of diffuseness parameter lie in the theoretically expected range between $1-2$.

\section{Precautions and Sources of Error}

\begin{enumerate}
    \item The sample not being properly in touch with the probes. and loose connections in any place of the circuit. improperly connected wires can increase the capacitance in the circuit.
    \item The variable resistor may saturate before the Curie temperature if a sample is taken with a high sample resistance, which will alter the predicted plot form.
    \item Readings beyond the lowest allowable voltage amplitude may result in variable resistance inaccuracy.
    \item Sample impurities that raise the curie temperature.
    \item The spring-loaded
    probes may introduce additional contact resistance, leading to an inaccurate measurement of capacitance. This occurs due to improper probe placement or oxidation of the
    sample’s conductive surfaces
    \item The accuracy of the capacitance measurement depends on precise balancing of the Schering
    Bridge.
    Small misalignments in tuning the
    variable capacitors and resistors can introduce
    systematic errors
    \item Slight variations
    in the thickness of the BaTiO$_3$ sample can introduce errors in the dielectric constant calculation.
    Since permittivity depends on the
    capacitance-to-thickness ratio, any deviation
    in thickness directly affects the computed values.
\end{enumerate}