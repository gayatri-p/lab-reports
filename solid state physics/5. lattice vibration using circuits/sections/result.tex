\vspace{-2em}
\section{Discussion \& Conclusion}

In this experiment, we explored the vibrational properties of
one-dimensional monoatomic and diatomic lattices using
an electrical circuit analogy. By constructing equivalent
LC circuits, we analyzed the dispersion relation by observing the phase change in the circuit using Lissajous figures, across a range of frequencies.

For the monoatomic lattice, the maximum theoretical and observed frequencies are,
\begin{align*}
    f_\text{max, th} &= (46.16 \pm 0.60) \text{ kHz}\\
    f_\text{max, obs} &= (50.68 \pm 0.01) \text{ kHz}
\end{align*}
This confirms the expected behavior of the system as a natural low-pass filter. The minor
deviation, which amounted to $\approx$9.7\%, can be attributed to component tolerances and measurement uncertainties.

For the diatomic lattice, the emergence of a frequency bandgap was observed, validating theoretical predictions. 
The cutoff frequencies for the acoustic and optical branches are,
\begin{align*}
    f_\text{acoustic, th} &= (18.52 \pm 0.03) \text{ kHz}\\
    f_\text{acoustic, obs} &= (20.13 \pm 0.01) \text{ kHz}\\
    f_\text{optical, th} &= (32.79 \pm 0.02) \text{ kHz}\\
    f_\text{optical, obs} &= (25.22 \pm 0.01) \text{ kHz}
\end{align*}

And the theoretical and observed frequency bandgaps are,

\begin{align*}
    f_\text{gap, th} &= (14.27 \pm 0.05) \text{ kHz}\\
    f_\text{gap, obs} &= (5.09 \pm 0.01) \text{ kHz}
\end{align*}

The deviation in $f_\text{gap}$ is much higher, around 64\%. We can see that this is because of the heavy deviation in the cutoff frequency of the optical branch. Errors in measurement and
practical component variations likely contributed to the discrepancies.

\section{Precautions and Sources of Error}

    \begin{enumerate}
        \item Ensure that all connections share a common ground
        to prevent signal interference.
        \item Avoid loose connections and ensure secure wiring
        to prevent fluctuations in circuit behavior.
        \item Verify that all inductors and capacitors remain
        within their specified tolerance limits to reduce de-
        viations from theoretical predictions.
        \item Handle circuit components carefully to prevent
        damage or changes in their electrical properties.
        \item Use properly calibrated measuring instruments
        such as oscilloscopes and multimeters to obtain pre-
        cise readings.
    \end{enumerate}