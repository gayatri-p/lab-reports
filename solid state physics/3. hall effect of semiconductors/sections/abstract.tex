\begin{abstract}
% variation of hall coefficient to the temperature of the semiconductor indicates that the mobility of the charge carries are effected by the 
In this experiment, we investigate the Hall effect of semiconductors and calculate the value of the
Hall coefficient for both n- and p-type Germanium and n-type Silicon samples at room temperature. In the second part of the experiment, we examine the dependence of the
Hall coefficient on temperature for p-type germanium, and observe an inversion in the Hall coefficient
at high temperatures. Furthermore, we find that the Hall coefficient exhibits a nonlinear decrease
with increasing temperature. These results have important implications for the understanding and
characterization of semiconductors and their electronic properties. Hence the Hall effect provides insights into properties like carrier type, mobility, and concentration, which conductivity measurements alone usually cannot reveal.
\end{abstract}