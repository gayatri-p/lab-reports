\section{Results}
My observing and measuring the hall effect of semiconductors, we were able to determine the Hall coefficient $R_H$, charge carrier density $n$ and the carrier mobility $\mu_e$ of three different semiconductors as follows,

\begin{itemize}
    \item Ge p-type
    \begin{align*}
         R_H &= (1.300 \pm 0.008) \times 10^4 \text{ cm}^3/\text{C}\\
         n &= (4.81 \pm 0.03) \times 10^{14} \text{ cm}^3\\
         \mu_e &= (1.86 \pm 0.26) \times 10^3 \text{ cm}^2/\text{Vs}
    \end{align*}
    \item Ge n-type
    \begin{align*}
         R_H &= -(2.154 \pm 0.006) \times 10^4 \text{ cm}^3/\text{C}\\
         n &= (2.90 \pm 0.01) \times 10^{14} \text{ cm}^3\\
         \mu_e &= (3.08 \pm 0.44) \times 10^3 \text{ cm}^2/\text{Vs}
    \end{align*}
    \item Si n-type
    \begin{align*}
         R_H &= -(1.791 \pm 0.018) \times 10^4 \text{ cm}^3/\text{C}\\
         n &= (3.49 \pm 0.03) \times 10^{14} \text{ cm}^3\\
         \mu_e &= (0.75 \pm 0.03) \times 10^3 \text{ cm}^2/\text{Vs}
    \end{align*}
\end{itemize}

\section{Conclusion}

In conclusion, our experiments for measuring the Hall coefficient have shown that the polarity of the Hall voltage is opposite for p-type and n-type samples, resulting in a change in the sign of $R_H$. We found that the calculated absolute value of $R_H$ for n-type germanium is higher than
that of p-type germanium, indicating a lower charge density in the former. This is because the absolute value of $R_H$ is inversely proportional to the charge density, which is in turn proportional to the number of charge carriers.
The carrier mobility of p-type semiconductor is also significantly higher than those for n-type semiconductors.

Similarly, we can also see that the carrier mobility is higher for n-type Ge than p-type Ge. This is expected behaviour as the mobility of electrons is greater than that of holes. Since conduction electrons travel in the conduction band and valence electrons (holes) travel in the valence band, the movement of valence electrons are restricted under an applied electric field. Thus, the effective mass of holes are larger and their time between scattering is lesser resulting in lesser mobility.  

Our temperature-dependent analysis revealed that the value of $R_H$ decreases with increasing temperature for the p-type sample and eventually becomes negative as stabilises, as expected due to an increase in the number of negative charge carriers with an increase in temperature. The decrease is not linear because of the significant difference between the mobilities of the two types of charge carriers at low temperatures.

\section{Precautions and Sources of Error}

The fluctuations in the values obtained could be attributed to several factors like,

    \begin{enumerate}
        \item Fluctuations in the coil current during the experiment, which may result in variations in the magnetic field strength experienced by the sample. Such
        fluctuations could be caused by external disturbances or instabilities in the power supply.
        \item Impurities in the sample, which could introduce additional sources of scattering and lead to variations
        in the measured value of $R_H$. Such impurities may
        be present in the bulk of the material or at the interfaces between different layers or regions of the
        sample.
        \item Thermal effects on the sample, caused by high probe
        current or non-zero thermal EMF. These effects
        could cause changes in the temperature of the sample, leading to variations in the measured value of
        $R_H$ . The magnitude of such effects may depend on
        the specific characteristics of the sample and the
        experimental setup.
        \item Variations in the ambient temperature of the room,
        which could affect the electrical properties of the sample and lead to variations in the measured value
        of $R_H$. Such variations could be due to fluctuations
        in air conditioning or heating systems, or other en
        vironmental factors.
        % \item High or variable resistance at the contacts between the Hall probe
        % and the sample can distort the current flow
        % and lead to incorrect voltage readings.\\
    \end{enumerate}

\noindent Note that other uncontrolled variables could also contribute to the observed variations in the measured values. More accurate methods like using the four-point probe technique or the Van der Pauw method eliminates errors from contact resistance, improving accuracy.