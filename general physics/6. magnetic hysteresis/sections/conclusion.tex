\section{Results and Discussion}
From the hysteresis curve of the ferromagnetic block, we have found out the following parameters.

\begin{itemize}
    \item Saturation magnetization, $B_S= (3650 \pm 1)$ Gauss
    \item Remanence, $B_r= (570 \pm 1)$ Gauss
    \item Coercivity, $H_C= (516 \pm 3)$ A/m\\
\end{itemize} 

Only ferromagnetic materials exhibit hysteresis
curves. The area under the curve represents the energy lost in one complete cycle during magnetization and demagnetization. Materials with greater retentivity are used as
permanent magnets.

We also we able to successgly degauss the iron core, i.e. remove any remnant magnetic field by applying a rapidly oscillating magnetic field.

\section{Precautions}
\begin{itemize}
    \item Avoid flow of large current in the coils for prolonged time.
    \item Avoid taking readings out of order.
\end{itemize} 