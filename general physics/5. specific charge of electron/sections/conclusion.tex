\section{Results}
The specific charge of an electron was estimated in this experiment using two methods. 

In the first method, the acceleration potential ($U$) of the electron beam was fixed and the current in the Helmholtz coil ($I$) was varied to get electron beams of different radii. The observed value of $e/m_e$ was found out to be,

\begin{align*}
    e/m_e = (1.744\pm0.012)\cross10^{11} \text{ C kg}^{-1}
\end{align*}

In the second method, the current in the Helmholtz coil was fixed and the acceleration potential of the electron beam was varied to get beams of different radii. The observed value of $e/m_e$ was found out to be,

\begin{align*}
    e/m_e = (1.410\pm0.121)\cross10^{11} \text{ C kg}^{-1}
\end{align*}

The mean value of the specific charge of an electron from both the methods comes out to be,
\begin{align*}
    e/m_e = (1.577\pm0.061)\cross10^{11} \text{ C kg}^{-1}
\end{align*}


\section{Conclusion}
The literature value of $e/m_e$ is 1.759$\cross10^{11}$ C kg$^{-1}$. The values obtained through experiment are close to this value, within some error bars. Since mass of an electron is difficult to measure directly, once the specific charge is calculated, it is easier to find the $m_e$ if the elementary charge $e$ is known. 
This also helps us understand how electrons move under electric field especially compared to other particles. Two particles withe the same specific charge will move in the path in a vacuum, when subjected to the same electric and magnetic fields.

\section{Precautions \& Sources of Error}
\begin{enumerate}
    \item Perform the experiment in a dark area and avoid parallax error while taking any readings.
    \item Make sure the connections are proper before switching on the electricity.
    \item Make sure that the electron beam is in fact perpendicular to the magnetic field.
    \item Do not exceed any current or voltage values as cautioned in the apparatus.
\end{enumerate}