\section{Error Analysis}
From eqn. (18), we can see that the error in Young's modulus (Y) will be propagated from the error values of measured quantities $b$, $d$, $l$, and slope ($m$) as follows:

\begin{equation*}
    \frac{\Delta Y_{ij}}{Y_{ij}} = \sqrt{\left(\frac{\Delta l}{l}\right)^2 + \left(\frac{\Delta b}{b}\right)^2 + \left(\frac{3\Delta d}{d}\right)^2 + \left(\frac{\Delta m_{ij}}{m_{ij}}\right)^2}
\end{equation*}

Using $\Delta l = 1$ mm, $\Delta b = 0.1$ mm and $\Delta d = 0.01$ mm, the calculated errors are $\Delta Y_{12} = 3.47$ GPa, $\Delta Y_{23} = 17.02$ GPa and $\Delta Y_{13} = 5.02$ GPa. Therefore, the average error will be measured by,

\begin{align*}
    \Delta Y &= \frac{1}{3}\sqrt{(\Delta Y_{12})^2 + (\Delta Y_{23})^2 + (\Delta Y_{13})^2} \\
    &= 6.03 \text{GPa}
\end{align*}

Similarly, from eqn (19), error in Poisson's ratio ($\Delta \sigma$) is the same as the error in slope of the $\rho_x$ vs $\rho_y$ plot.\\
Thus the average error value in Poisson's ratio can be written as,

\begin{align*}
    \Delta \sigma &= \frac{1}{3}\sqrt{(\Delta \sigma_{200})^2 + (\Delta \sigma_{250})^2 + (\Delta \sigma_{300})^2} \\
    &= 0.007
\end{align*}