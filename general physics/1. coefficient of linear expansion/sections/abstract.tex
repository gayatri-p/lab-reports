\begin{abstract}
    In our everyday life, all of us have come across the expansion of metals upon heating - a quite familiar example would
be the expansion of mercury in a thermometer on rising temperature. In this report, we will be studying the linear thermal expansion of brass and will determine its coefficient of thermal expansion. We observe that for small
changes in temperature, the change in length of a metal is proportional to its original length and the change in temperature - the consant of proportionaity involved here is called the coefficient of thermal expansion($\alpha$). We will be
using Fizeau's method - an optical method - to find the change in length of the metal bar and using that, we determine $\alpha$.
\end{abstract}