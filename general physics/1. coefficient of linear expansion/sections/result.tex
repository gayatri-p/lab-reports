\section{Results and Conclusion}

From the experiment, the coefficient of thermal expansion of Brass was measured to be,

\begin{align*}
    \alpha = (19.114 \pm 0.348) \times 10^{-3}\,\text{K}^{-1}
\end{align*}

We found that the $\theta$ vs $T$ gives us a nearly straight-line
plot, as predicted. The standard value of the coefficient of thermal expansion of Brass is,

\begin{align*}
    \alpha_\text{std} = 20.3 \times 10^{-3}\,\text{K}^{-1}
\end{align*}

Hence, there is a 5.8\% deviation from the standard value. This could be due to various reasons including 

\begin{enumerate}[label=(\roman*)]
    \item disturbance of the apparatus which can cause the
    optical paths to be disturbed hence altering the
    fringe widths
    \item missing fringes during the measurement
    \item the apparatus in which the brass rod is kept can
    expand on heating and hence alters the wedge angle
    and thereby the fringe width.
\end{enumerate}

\section{Precautions}

\begin{enumerate}
    \item Make sure that the fringes are vertical and they remain
    vertical even after the plate is lifted for some wedge angle.
    If not, adjust the glass plates so that you obtain
    vertical fringes.
    \item Do not disturb the apparatus while taking measure-
    ments.
    \item Handle the glass plates carefully without contaminating
    its surface.
\end{enumerate}