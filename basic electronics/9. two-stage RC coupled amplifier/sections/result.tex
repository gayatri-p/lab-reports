\section{Results and Discussion}
We have successfully constructed a two stage RC coupled transistor amplifier circuit, and anaysed its frequency response curve. We found that the gain is maximum and stable in the mid-frequency range. The bandwidth of the frequency response came out to be from 48.46 Hz to 712.89 kHz.

Using data from the D.C. analysis of the circuit, the following values were calculated:

\begin{itemize}
    \item Input Impedance of Stage 1
        \begin{align*}
            Z_{i1} = (2.083 \pm 0.001)\,\text{k}\Omega
        \end{align*}
    \item Output Impedance of Stage 1
        \begin{align*}
            Z_{i1} = (3.835 \pm 0.001)\,\text{k}\Omega
        \end{align*}
    \item Input Impedance of Stage 2
        \begin{align*}
            Z_{i2} = (2.062 \pm 0.001)\,\text{k}\Omega
        \end{align*}
    \item Output Impedance of Stage 2
        \begin{align*}
            Z_{o2} = (3.843 \pm 0.001)\,\text{k}\Omega
        \end{align*}
\end{itemize}

The theoretical value of the overall mid-frequency gain voltage gain was calculated as (14.48 $\pm$ 0.07) (or 23.21 $\pm$ 0.04 dB). The observed value of mid-frequency gain is 19.75 (or 25.91 dB). We can see that overall gain is close to the product of the voltage gain of both the individual stages. Hence, we have achived higher overall voltage gain by connecting two stages of an RC coupled transistor amplifier circuit.

The deviation in the values could be due to (i) error in measurement of the resistance/capacitance values, or (ii) the extra resistance offered by the connecting wires, or (iii) fluctuation in the source voltage.

\section{Precautions}
\begin{enumerate}
    \item Vary the input signal frequency slowly.
    \item Connect electrolytic capacitors carefully.
\end{enumerate}