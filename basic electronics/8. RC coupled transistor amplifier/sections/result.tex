\section{Error Analysis}
From Eq. (9), we can find the error in the theoretical value of $A_V$ from,

\begin{align*}
    \Delta A_V &= \sqrt{\left( \frac{\partial A_V}{\partial R_C} \Delta R_C \right)^2 + \left( \frac{\partial A_V}{\partial R_{E1}} \Delta R_{E1} \right)^2 + \left( \frac{\partial A_V}{\partial r_e} \Delta r_e \right)^2} \\ 
    & = 0.004
\end{align*}

Using $\Delta R_C = 0.001$ k$\Omega$, $\Delta R_{E1} = 0.1$ $\Omega$ and $\Delta I_E = 0.01$ mA, where $r_e=26$ mV$/I_E$.

\section{Results and Discussion}
We have successfully constructed a RC coupled common emitter transistor amplifier circuit, and anaysed its frequency response curve. We found that the gain is maximum and stable in the mid-frequency range. The bandwidth of the frequency response came out to be from 25.58 Hz to 784.16 kHz.

Using data from the D.C. analysis of the circuit, the input and output impedance of the circuit was calculated as $Z_i=$ 1.941 k$\Omega$ and $Z_o=$ 3.849 k$\Omega$.

Similarly, the theoretical value of the mid-frequency gain was calculated as 7.861 $\pm$ 0.004 (or 17.91 dB). The observed value of mid-frequency gain is 7.8 (or 17.84 dB).

\section{Precautions}
\begin{enumerate}
    \item Vary the input signal frequency slowly.
    \item Connect electrolytic capacitors carefully.
\end{enumerate}