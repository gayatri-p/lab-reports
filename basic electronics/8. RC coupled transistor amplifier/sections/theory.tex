\section{Objective}

To design a common emitter transistor (NPN) amplifier circuit, and to obtain the frequency response curve of the amplifier and to determine the mid-frequency gain, $A_\text{mid}$, lower and higher cutoff frequency of the amplifier circuit.

\section{Theory}
The most common circuit configuration for an NPN transistor is that of the Common
Emitter Amplifier and that a family of curves known commonly as the Output
Characteristics Curves, relates the Collector current ($I_C$), to the output or Collector
voltage ($V_{CE}$), for different values of Base current ($I_B$). All types of transistor amplifiers
operate using AC signal inputs which alternate between a positive value and a negative
value. Presetting the amplifier circuit to operate between these two maximum or peak
values is achieved using a process known as Biasing. Biasing is very important in
amplifier design as it establishes the correct operating point of the transistor amplifier
ready to receive signals, thereby reducing any distortion to the output signal.
The single stage common emitter amplifier circuit shown below uses what is commonly
called "Voltage Divider Biasing". The Base voltage ($V_B$) can be easily calculated using
the simple voltage divider formula below:
\begin{align}
    V+B = \frac{V_{CC}R_2}{R_1+R_2}
\end{align}
Thus the base voltage is fixed by biasing and it is independent of base current as long as
the current in the divider circuit is large compared to the base current. Thus assuming $I_B \approx 0$, one can do the approximate analysis of the voltage divider network without using the
transistor gain, $\beta$, in the calculation. Note that the approximate approach can be applied
with a high degree of accuracy when the following condition is satisfied:
\begin{align}
    \beta R_E = \ge 10R_2
\end{align}

\subsection*{Load line and Q-point}

A static or DC load line can be drawn onto the output characteristics curves of the
transistor to show all the possible operating points of the transistor from fully "ON" ($I_C$ =
$V_{CC}$/($R_C$ + $R_E$)) to fully "OFF" ($I_C$ = 0). The quiescent operating point or Q-point is a
point on this load line which represents the values of $I_C$ and $V_{CE}$ that exist in the circuit
when no input signal is applied. Knowing $V_B$, $I_C$ and $V_{CE}$ can be calculated to locate the
operating point of the circuit as follows:
\begin{align}

\end{align}
So, the emitter current,
\begin{align}

\end{align}
It can be noted here that the sequence of calculation does not need the knowledge of $\beta$
and $I_B$ is not calculated. So the Q-point is stable against any replacement of the transistor.
Since the aim of any small signal amplifier is to generate an amplified input signal at the
output with minimum distortion possible, the best position for this Q-point is as close to
the centre position of the load line as reasonably possible, thereby producing a Class A
type amplifier operation, i.e. $V_{CE} = 1/2V_{CC}$.

\subsection*{Coupling and Bypass Capacitors}

In CE amplifier circuits, capacitors $C_1$ and $C_2$ are used as Coupling Capacitors to separate
the AC signals from the DC biasing voltage. The capacitors will only pass AC signals
and block any DC component. Thus they allow coupling of the AC signal into an
amplifier stage without disturbing its Q point. The output AC signal is then superimposed
on the biasing of the following stages. Also a bypass capacitor, CE is included in the
Emitter leg circuit. This capacitor is an open circuit component for DC bias, meaning that
the biasing currents and voltages are not affected by the addition of the capacitor
maintaining a good Q-point stability. However, this bypass capacitor acts as a short
circuit path across the emitter resistor at high frequency signals increasing the voltage
gain to its maximum. Generally, the value of the bypass capacitor, CE is chosen to
provide a reactance of at most, 1/10th the value of $R_E$ at the lowest operating signal
frequency.

\subsection*{Amplifier Operation}

Once the Q-point is fixed through DC bias, an AC signal is applied at the input using
coupling capacitor $C_1$. During positive half cycle of the signal $V_B$E increases leading to
increased $I_B$. Therefore $I_C$ increases by $\beta$ times leading to decrease in the output voltage, $V_{CE}$. Thus the CE amplifier produces an amplified output with a phase reversal. The
voltage Gain of the common emitter amplifier is equal to the ratio of the change in the
output voltage to the change in the input voltage. Thus,
\begin{align}

\end{align}
The input (Zi) and output (Zo) impedances of the circuit can be computed for the case
when the emitter resistor $R_E$ is completely bypassed by the capacitor, CE:
Zi = R1 ║R2║$\beta$re and Zo = $R_C$║ro
where re (26mV/IE) and ro are the emitter diode resistance and output dynamic resistance
(can be determined from output characteristics of transistor). Usually ro≥10 $R_C$, thus the
gain can be approximated as
\begin{align}

\end{align}
The negative sign accounts for the phase reversal at the output.
Note: In the circuit diagram provided below, the emitter resistor is split into two in order
to reduce the gain to avoid distortion. So the expression for gain is modified as
\begin{align}

\end{align}
Frequency Response Curve
The performance of an amplifier is characterized by its frequency response curve that
shows output amplitude (or, more often, voltage gain) plotted versus frequency (often in
log scale). Typical plot of the voltage gain of an amplifier versus frequency is shown in
the figure below. The frequency response of an amplifier can be divided into three
frequency ranges.
4
The frequency response begins with the lower frequency range designated between 0 Hz
and lower cutoff frequency. At lower cutoff frequency, fL , the gain is equal to 0.707 Amid.
Amid is a constant mid-band gain obtained from the mid-frequency range. The third, the
higher frequency range covers frequency between upper cutoff frequency and above.
Similarly, at higher cutoff frequency, fH, the gain is equal to 0.707 Amid. Beyond this the
gain decreases with frequency increases and dies off eventually.
The Lower Frequency Range
Since the impedance of coupling capacitors increases as frequency decreases, the voltage
gain of a BJT amplifier decreases as frequency decreases. At very low frequencies, the
capacitive reactance of the coupling capacitors may become large enough to drop some
of the input voltage or output voltage. Also, the emitter-bypass capacitor may become
large enough so that it no longer shorts the emitter resistor to ground.
The Higher Frequency Range
The capacitive reactance of a capacitor decreases as frequency increases. This can lead to
problems for amplifiers used for high-frequency amplification. The ultimate high cutoff
frequency of an amplifier is determined by the physical capacitances associated with
every component and of the physical wiring. Transistors have internal capacitances that
shunt signal paths thus reducing the gain. The high cutoff frequency is related to a shunt
time constant formed by resistances and capacitances associated with a node.
Design:
Before designing the circuit, one needs to know the circuit requirement or specifications.
The circuit is normally biased for $V_{CE}$ at the mid-point of load line with a specified
collector current. Also, one needs to know the value of supply voltage $V_{CC}$ and the range
of $\beta$ for the transistor being used (available in the datasheet of the transistor).
Here the following specifications are used to design the amplifier:
$V_{CC}$ = 12V and $I_C$ = 1 mA
Start by making VE= 0.1 $V_{CC}$. Then $R_E$ = VE/IE (Use IE≈$I_C$).
Since $V_{CE}$ = 0.5 $V_{CC}$, Voltage across $R_C$ = 0.4$V_{CC}$, i.e. $R_C$ = 4.$R_E$
In order that the approximation analysis can be applied,
\begin{align*}

\end{align*}
. Here  is the
minimum rated value in the specified range provided by the datasheet (in this
case  =50).
Finally,
2
2
1
1
R
V
V
R 
, V1 (= $V_{CC}$-V2)and V2 (= VE+$V_B$E) are voltages across R1 and R2,
respectively.
Based on these guidelines the components are estimated and the nearest commercially
available values are used.