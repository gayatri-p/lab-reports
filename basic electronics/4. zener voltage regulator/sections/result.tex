\section{Results and Discussion}
    \paragraph*{\textbf{Load Regulation:}} For an unregulated D.C. voltage of $\sim$11.6 V, we have observed that the load voltage also increases until a certain $R_L$, after which it becomes roughly constant, i.e. regulated. 
        
        By measuring $V_L$ at full and no load current situations, we have calculated the percentage load regulation in the circuit as (1.366 $\pm$ 0.024) \%.

    \paragraph*{\textbf{Line Regulation:}} We have observed that as the load input voltage increases, load voltage also increases until it becomes roughly constant after $V_{TH}$ crosses the zener breakdown. This is when the zener diode becomes active.

        Hence, by measuring $V_L$ at at high and low input lines (when zener is activated), we have calculated the percentage line regulation as (1.617 $\pm$ 0.017) \%.\\

        Typical well-regulated power supplies have load regulations of less than 1\%, meaning that the output voltage will change by a maximum of 1\% over the supply’s load current range. The values we have achieved are quite close to well-regulated.

        Hence we have succesfully constructed a power suppy using zener diode as the voltage regulator.

\section{Precautions}
\begin{enumerate}
    \item The transformer must be handled carefully.
    \item Switch on the circuit only after verifying the connections to be proper.
    \item Do not change the resistor or the capacitor while the circuit is switched on.
\end{enumerate}