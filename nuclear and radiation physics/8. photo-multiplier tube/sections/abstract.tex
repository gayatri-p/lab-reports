\begin{abstract}
    This study explores the fundamental characteristics of PMTs, focusing on their high amplification capability, spectral response, and intrinsic dark current. Using a Hamamatsu H9305 03 PMT module, we conduct an experimental analysis of gain dependence on control voltage, spectral sensitivity across the 400–700 nm range, and variations in dark current as a function of supply voltage. Our findings highlight the signal amplification properties of the PMT, its wavelength dependent detection efficiency, and the influence of dark current on measurement precision.  
\end{abstract}