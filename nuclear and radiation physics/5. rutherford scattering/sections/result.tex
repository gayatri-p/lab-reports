\section{Discussion \& Conclusion}
We have successfully performed Rutherford's scattering experiment. In the first part, we by varying the angle between the detector and the emitter, we were able to get the count rate distribution as a function of the scattering angle. Our distribution more or less agrees with the general form of Rutherford scattering as given by Eq. \ref{eq:gen}.
The fitting parameters obtained were:


The non-zero offsets B suggests inaccuracies stemming from misalignment in the collimator slit and minor horizontal angular displacement errors. From these, an angular displacement error of $(0.026 \pm 0.014)^\circ$ was identified, Since the resolution of the instrument is only $5^\circ$, this was considered almost negligible. Despite these limitations, we able to successfully confirm the $\sin^{-4}(\theta/2)$ dependence in the distribution from graphical analysis.

In the second part of the experiment, we performed Rutherford’s scattering experiment at a fixed scattering angle for both gold and aluminium foils. Then, by Eq. \ref{eq:part2}, we were able to estimate the nuclear charge of Aluminium as,

\begin{align*}
    \boxed{Z_\text{Al} = 12.6 \pm 1.8}
\end{align*}

The high amount of error the measurement is due to the extreme variation in count number for both foils. Ideally, these measurements should be taken for longer periods of time to minimise the error. Since the true value of $Z_\text{Al}$ is 13, our results are within the acceptable values, with a percentage error of $3.07$\%.

\section{Precautions and Sources of Error}

    \begin{enumerate}
        \item When the vacuum pump is running, the gold leaf
        needs to be parallel to the nozzle. High velocity air
        might rupture the thin \& expensive gold foil.
        \item  The gold foil should not be touched by the hands.
        The oil from the hands might be deposited on the
        foil, which will cause the readings to be inaccurate.
        It can also cause the foil to be damaged.
        \item Any amount of air inside the vacuum chamber will
        cause the readings to be non-uniform and inaccurate.
        \item The offset discriminator has to be set properly to
        get the correct readings.
    \end{enumerate}

