\section{Discussion \& Conclusion}

We have successfully conducted an experiment analysing Compton scattering of electrons.

We first calibrated the MCA setup using a mixed source of Americum and Caesium. Then, using different scattering bodies we observed the compton effect using a Cs-137 source.

The observed data in our experiment shows that the intensity of scattered photons decreases as the scattering angle increases. This observation agrees with the Compton formula, which describes the energy transfer from photons to electrons in the scattering process.
We also found that the energy of scattered photons decreases as the scattering angle increases, which again agrees with the Compton formula. This is because the energy of scattered photons is related to the energy lost by the electrons in the scattering process.

Our experiment also revealed that the detector used in the experiment had different calibration factors for different materials. The calibration factors obtained are:\\

\begin{itemize}
    \item For Aluminium: $2.11\times 10^{31} \pm 0.02$ m$^{-2}$
    \item For Copper: $1.57 \times 10^{31} \pm 0.02$ m$^{-2}$\\
\end{itemize}

This implies that for particular cross-sections, the relative intensity for cooper is more than for aluminium.

By measuring the scattered photon energy and angle, along with the incident photon energy, it is possible to calculate the rest mass of the electron. This technique is widely used in experimental physics to measure the masses of particles.

\section{Precautions and Sources of Error}

\begin{enumerate}
    \item Properly shield the setup before taking data.
    \item Measure the relative intensities using the software by only taking the appropriate energy ranges.
\end{enumerate}