\section{Discussion \& Conclusion}
In the experiment, we investigated the operation of an ADC and DAC amplifier and built them with simple electronic components. We discovered the linearity error in the slope of the DAC plot between the theoretical and experimental values using our understanding of the numerous parameters governing the performance of a converter and discovered that it agreed more for larger number of bits at the digital input, but the difference iwas negligible. We noticed that while the DAC converter is operational, it displays a tiny deviation from the required voltage, which may be ignored. However, with larger bit converters, the variation is seen to decrease. These variations could be due to the assumption of $R=1$ k$\Omega$ in our calculations, which may not be the case at all times. The resistors we used in the experiment had a tolerance rate upto 20\%.

We have also seen how the precision of the setup increases with the increase in number of bits as seen in the difference in successive voltage values in 3-bit and 4-bit DAC, which were on average $1.27$ V and $0.64$ V respectively. 

We also learnt how to convert digital output to decimal using a BCD display. 
Hence, these experiments highlight the importance of accurate component selection and proper circuit design in ensuring reliable and efficient signal conversion.

\section{Precautions and Sources of Error}

    \begin{enumerate}
        \item To avoid loose connections and short circuits, the
        connections should be correctly constructed.
        \item Resistors and integrated circuits must be tested for characteristics and functionality before being used
        in a circuit.
        \item To avoid fusing, connect the LEDs to the resistor.
        \item To avoid burning an op-amp, the biassing voltage
        should not exceed the specified value.
    \end{enumerate}

