\section{Discussion \& Conclusion}

The operating voltage of the scintillation detector was found to be 500V, where the detector exhibits a resolution of
$(9.14 \pm 0.92)$\%. Using this we successfully analysed the energy spectrum of Cs-137 using SCA.
The error in the Gaussian curve fitting to find resolution could be reduced
further by increasing the number of data points.
Now, by calibrating the MCA channels with
known gamma sources, we were able to accurately
determine the gamma energy peaks of Na-22, which came out to be $(511.80 \pm 3.95)$ keV and $(1276.35 \pm 6.54)$ keV. These are very close to the literature values of 511 keV and 1274.5 keV respectively.
We then calculated the mass absorption coefficient of aluminium to be $(0.076 \pm 0.011)$ g/cm$^2$. Hence, it is an efficient material for shielding against gamma radiation.

\section{Precautions and Sources of Error}  

    \begin{enumerate}
        \item Inaccurate energy or efficiency calibration could impacts results. So, accurately calibrate
        the detector and MCA before measurements.
        \item Voltage or temperature fluctuations lead to calibration drift, hence monitor the temperature carefully.
        \item Contamination in the sample could lead to incorrect data.
        \item Make sure to properly shield the radioactive source when not in use. Use proper handling tools to avoid any accidents.
        \item Background radiation could cause incorrect data, so properly shield the setup.
    \end{enumerate}
