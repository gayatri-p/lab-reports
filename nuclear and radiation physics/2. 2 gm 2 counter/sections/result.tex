\section{Discussion \& Conclusion}

Using the half-thickness method, the range of $\beta$ particles from Sr-90 through an Aluminium absorber was measured to be $(0.826 \pm 0.060)$ mg/cm$^2$. The end point energy was measured to be $(0.761\pm 0.128)$ MeV. This was obtained by comparing the range of $\beta$ particles from the Sr-90 source with a Tl-204 source whose end-point energy is known to us.

By studying the attenuation of Bremsstrahlung radiation by different combinations of Copper, Aluminium and Perspex, we found that more attenuation occurs when Perspex is facing the source. Conversely, we observed higher Bremsstrahlung counts when metals are facing the source. Among the metals, Copper was observed to show more counts than Aluminium, in line with the theory that the amount of Bremsstrahlung increases with the atomic number of the absorbing material.

We also studied backscattering, and observed that the count rate initially increases with an increase in scattering body thickness and eventually reaches saturation. The saturation thickness of Aluminium in our case was observed to be $(0.35 \pm 0.05)$ mm with a Sr-90 $\beta$-source. 

\section{Precautions and Sources of Error}

    \begin{enumerate}
        \item The \textit{dead time} of the GM counter, which is the time during which it cannot detect another event, might have led to a reduction in the count rate and an overestimation of the sample’s activity.
        \item Electronic noise produced by the components of the GM counter might have led to an increase in the count rate and an overestimation of the sample’s activity.
        \item We observe a high amount of fluctuation due to background noise, so to obtain accurate counts, a large number of observations are required.
        \item The beta particles from the sample might have been absorbed by the GM tube’s window, or the air between the sample and the GM counter. 
        \item Fluctuations in the GM counter’s operating voltage might have led to changes in the count rate and an overestimation or underestimation of the sample’s activity.
    \end{enumerate}