\section{Procedure}

\subsection{Calibration of Lock-in Amplifier}

\begin{enumerate}
    \item For calibrating the Lock-In amplifier, a voltage
    divider circuit using 4.7k$\Omega$ and 12$\Omega$ is created.
    \item The reference is taken across former and the
    input signal to the amplifier is taken from the
    later. The DC output from the amplifier is measured using a digital multimeter.
    \item The DC offset
    is adjusted and the phase between the two signals is set to zero using the phase nob of the
    amplifier.
    \item For every frequency between 300 to
    1500 Hz at an interval of 300 Hz values are taken
    (Table 1 and Table 3) for two values of gain of 50 and 100.
\end{enumerate}

\subsection{Calculation of Mutual Inductance}
\begin{enumerate}
    \item For finding the mutual inductance of the input
    signal is taken from the secondary coil.
    \item The
    voltage across the primary signal is taken as
    reference.
    \item The REF and REF’ are connected
    to oscilloscope to obtain the Lissajous figure to
    ensure that phase difference between them is 90
    degrees as a result the reference and the signal
    are in phase.
    \item Vary $V_{ac}$ from 7 - 15 V and note down the values of $V_{dc}$.
    \item $V_{ac}$ and $V_{dc}$ are measured for 4 sets of frequencies from 600 Hz to 1500 Hz.
\end{enumerate}

\subsection{Measurement of Low Resistance}

\begin{enumerate}
    \item With the same voltage divider circuit we used for calibration, the 4.7 k$\Omega$ is swapped with a 500 $\Omega$ resistor (across the reference).
    \item The input signal to the amplifier is connected across the test resistance.
    \item Vary $V_{ac}$ from 1 - 3 V and note down the values of $V_{dc}$.
    \item $V_{ac}$ and $V_{dc}$ are measured for 5 sets of frequencies from 300 Hz to 1500 Hz.\\
\end{enumerate}

