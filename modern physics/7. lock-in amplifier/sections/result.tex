\section{Results \& Discussion}
In this experiment, we studied a Lock-in Amplifier and used it to measure quantities the Mutual Inductance and low resistance values. The results are detailed below.

Firstly, we calibrated the Lock-in Amplifier for two values of gain at 50 and 100 to find the amplification factor, which comes out to be,

\begin{align*}
    \mu_{50} = 158.8 \pm 3.6\\
    \mu_{100} = 308.6 \pm 5.5
\end{align*}

The mutual inductance measured for the given coils comes out to be,

\begin{align*}
    M = (168.3 \pm 9.5)\,\mu\text{H}
\end{align*}

The value of low resistance measured comes out to be,

\begin{align*}
    r = (0.380 \pm 0.018)\,\Omega
\end{align*}


Hence we have found out that a lock-in amplifier plays a critical
role of phase-sensitive detection in improving
the signal-to-noise ratio. We can summarise the pros and cons of a lock-in amplifier as follows.

\subsection*{Advantages}
The stability of the lockin amplifier’s output in noisy environments,
thanks to the phase-locked loop, highlights the
lock-in amplifier’s strength in rejecting noise
and maintaining synchronization.
This is especially useful in environments where external noise sources could otherwise interfere with measurements, ensuring reliable data acquisition.

The clear linear relationship between the induced EMF and the current through the primary coil that you observed
is consistent with the theoretical model of mutual inductance. The lock-in amplifier’s ability
to measure this relationship accurately, even at
low signal levels, demonstrates its effectiveness
in detecting small signals in experiments with
limited signal strength.

\subsection*{Disadvantages}
The observation that
slight phase misalignments can significantly affect the output underlines the need for precise phase control when using lock-in amplifiers.
This precision in phase alignment is crucial for
accurate signal recovery, especially when dealing with weak signals.

The finite bandwidth of the lock-in amplifier is indeed a potential limitation, especially for signals with broad
frequency spectra. This restricts the range of
frequencies that can be effectively analyzed and
may require careful consideration when designing experiments or selecting equipment for a
broader range of frequencies.\\

Its also observed that the mutual inductance EMF
is characterized by being 90 degrees out of phase
with the current in the primary coil. It is directly
proportional to both the current in the primary coil
and the frequency of the applied signal.
These observations underscore both the strengths
and limitations of the lock-in amplifier, making it
a valuable tool for experiments involving weak signals and phase-sensitive measurements.
However,
its effectiveness depends on the frequency range and
phase alignment.


\section{Precautions and Sources of Error}

    \begin{enumerate}
        \item External electrical noise, thermal noise, or elec        tromagnetic interference can affect the input
        signal, adding unwanted components.
        \item The lock-in amplifier relies on phase-sensitive
        detection, so any misalignment between the ref        erence signal and the input signal’s phase will
        cause inaccurate signal extraction
        \item The low-pass filter in the lock-in amplifier, used
        to smooth the output signal, can introduce er        rors if not correctly configured.
        If the filter’s time constant is too short, it may pass noise; if
        too long, it may attenuate the desired signal.
        \item Fluctuations in the amplitude of the input sig        nal or reference signal can lead to variations in
        the measured output.
        \item The lock-in amplifier has a limited bandwidth,
        meaning it can only effectively analyze signals
        within a specific frequency range.
        \item If the input signal contains harmonics (multiples
        of the reference frequency), the lock-in amplifier
        may detect signals at these harmonic frequen        cies.
        \item A drift in the reference signal’s frequency or
        phase can desynchronize the phase-locked loop
        (PLL), which may lead to measurement errors
        over time.
    \end{enumerate}