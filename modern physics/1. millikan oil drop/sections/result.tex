\section{Results \& Discussion}

In the dynamic method, the minimum value of $ne$ measured came out to be $2.20 \times 10^{-19}$ C. Using this the mean value of $e$ was measured to be,
\begin{align*}
    e = (2.13 \pm 0.13) \times 10^{-19}\,\,C
\end{align*}

In the balancing method, the minimum value of $ne$ measured came out to be $2.27 \times 10^{-19}$ C. Using this the mean value of $e$ was measured to be,
\begin{align*}
    e = (2.42 \pm 0.31) \times 10^{-19}\,\,C
\end{align*}

The standard value of $e$ is $1.6 \times 10^{-19}$ C. Our measured values are 33\% and 51\% higher for each method respectively.
As discussed above, since the values are systematically deviated, there could be systematic errors present in the experiment. 
Despite that, our values are precise based on the error bars we obtained, especially in the case of dynamic method.

\section{Precautions}

    \begin{enumerate}
        \item Maintain a constant temperature in the experiment
        setup to prevent changes in air viscosity, which can
        alter the droplet’s behavior.
        \item Use a stable and precise voltage source for the electric field.
        Fluctuations in the voltage can lead to
        inaccurate measurements of the droplet’s charge.
        \item Ensure the oil droplets are small enough to be influenced by the electric field but large enough to be
        visible under the microscope. Accurate focusing is
        crucial for precise measurements.
        \item Use an accurate stopwatch or timing device to measure the time intervals as the droplet moves, which
        is critical for calculating the charge.
        \item Ensure that the chamber, plates, and all equipment
        are free of dust and contaminants to avoid interference with the motion of the oil droplets.
    \end{enumerate}

\section{Conclusion}
The Millikan oil drop experiment successfully measured
the elementary charge $e$, and provided robust evidence
for the quantization of electric charge. By
balancing the gravitational, buoyant, and electric forces
acting on tiny oil droplets, we were able to determine of the charge on individual droplets, and hence the charge of the electron.

Historically, the Millikan oil drop experiment significantly contributed to the development of quantum theory by providing direct evidence for the quantization of electric
charge.
Before this experiment, the idea that certain
physical properties, like charge, existed in discrete units
was still a hypothesis. Millikan’s work demonstrated that
electric charge is not continuous but rather comes in indivisible packets, specifically multiples of the elementary
charge ($e$). This finding was crucial in validating the concept of quantization, which is a cornerstone of quantum
mechanics.

The precise determination of the elementary charge is crucial for calculations in electromagnetism,
such as those involving Coulomb’s law, and for determining other fundamental constants, such as the
fine-structure constant. Understanding the quantization of charge also has been
essential in developing technologies such as semiconductors, transistors, and various electronic components, which rely on the controlled movement of
charge.