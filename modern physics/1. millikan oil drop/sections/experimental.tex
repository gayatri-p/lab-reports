\section{Observation and Calculations}

\begin{itemize}
    \item Distance between parallel plates, d $=5\times 10^{-3}$ m
    \item Distance where the velocity is measured, L $=1\times 10^{-3}$ m
    \item Density of oil, $\rho=929\text{ kg m}^{-3}$
    \item Density of air, $\rho_a=1\text{ kg m}^{-3}$
    \item Room temperature, $T=25^{\circ}\text{ C}$
    \item Atmospheric Pressure, $P=1\text{ atm}$
    \item Coefficient of viscosity of air, $\eta=1.8480 \times 10^{-5}$ kg/m.sec 
    \item For calculation purposes,
        \begin{align}
            C &= \frac{4}{3}\pi dg\,(\rho-\rho_a)=190.13\\
            D &= \frac{9\eta}{2g}\pi dg\,(\rho-\rho_a) = 9.04 \times 10^{-9}\\
            \zeta = c / 2P &= 4.06 \times 10^{-8}\\
            \text{where } c &= 6.17 \times 10^{-8} \text{ m of Hg m} \nonumber
        \end{align}
\end{itemize}

Tables \ref{obs1} and \ref{obs2} show the observational data and calculation of $ne$ for dynamic and balancing methods respectively using Eqns. 6 and 12. For simplification, we used
\begin{align*}
    \eta &= Dv_f\\
    ne = \frac{CTr^3}{V} &\text{ or }ne=\frac{Cr^3}{V_b}
\end{align*}

% Please add the following required packages to your document preamble:
% \usepackage{multirow}
% \usepackage{graphicx}
\begin{table*}[]
    \centering
    \resizebox{\textwidth}{!}{%
    \begin{tabular}{|c|c|c|c|c|c|c|c|c|c|c|c|}
    \hline
    Drop no. & Sl no. & Free Fall  & Rise time& Mean Free fall  & Mean Rise time  & Mean free fall velocity & Voltage & $\eta$  & r & T & $ne$ \\
     & & time (s) & (s) & time $t_f$ (s) & time $t_r$ (s) & ($v_f$) (in mm/s) & (V) & ($\times   10^{-12}$) & ($\times   10^{-7}$) (in m) & $(1+t_f/t_r)$ & ($\times 10^{-19}$ C) \\ \hline
    \multirow{5}{*}{1} & 1 & 10.20 & 2.40 & \multirow{5}{*}{9.36} & \multirow{5}{*}{2.54} & \multirow{5}{*}{0.11} & \multirow{5}{*}{145} & \multirow{5}{*}{0.97} & \multirow{5}{*}{9.43} & \multirow{5}{*}{4.69} & \multirow{5}{*}{51.51} \\ \cline{2-4}
     & 2 & 9.10 & 2.70 &  &  &  &  &  &  &  &  \\ \cline{2-4}
     & 3 & 9.10 & 2.70 &  &  &  &  &  &  &  &  \\ \cline{2-4}
     & 4 & 9.20 & 2.30 &  &  &  &  &  &  &  &  \\ \cline{2-4}
     & 5 & 9.20 & 2.60 &  &  &  &  &  &  &  &  \\ \hline
    \multirow{5}{*}{2} & 1 & 3.30 & 3.90 & \multirow{5}{*}{2.92} & \multirow{5}{*}{3.82} & \multirow{5}{*}{0.34} & \multirow{5}{*}{282} & \multirow{5}{*}{3.10} & \multirow{5}{*}{17.19} & \multirow{5}{*}{1.76} & \multirow{5}{*}{60.47} \\ \cline{2-4}
     & 2 & 2.80 & 3.80 &  &  &  &  &  &  &  &  \\ \cline{2-4}
     & 3 & 2.90 & 3.60 &  &  &  &  &  &  &  &  \\ \cline{2-4}
     & 4 & 2.60 & 3.80 &  &  &  &  &  &  &  &  \\ \cline{2-4}
     & 5 & 3.00 & 4.00 &  &  &  &  &  &  &  &  \\ \hline
    \multirow{5}{*}{3} & 1 & 14.10 & 6.30 & \multirow{5}{*}{14.36} & \multirow{5}{*}{6.32} & \multirow{5}{*}{0.07} & \multirow{5}{*}{320} & \multirow{5}{*}{0.63} & \multirow{5}{*}{7.54} & \multirow{5}{*}{3.27} & \multirow{5}{*}{8.33} \\ \cline{2-4}
     & 2 & 15.30 & 6.30 &  &  &  &  &  &  &  &  \\ \cline{2-4}
     & 3 & 14.20 & 6.10 &  &  &  &  &  &  &  &  \\ \cline{2-4}
     & 4 & 14.40 & 6.50 &  &  &  &  &  &  &  &  \\ \cline{2-4}
     & 5 & 13.80 & 6.40 &  &  &  &  &  &  &  &  \\ \hline
    \multirow{5}{*}{4} & 1 & 16.30 & 3.91 & \multirow{5}{*}{16.998} & \multirow{5}{*}{3.816} & \multirow{5}{*}{0.06} & \multirow{5}{*}{419} & \multirow{5}{*}{0.53} & \multirow{5}{*}{6.90} & \multirow{5}{*}{5.45} & \multirow{5}{*}{8.12} \\ \cline{2-4}
     & 2 & 15.79 & 3.98 &  &  &  &  &  &  &  &  \\ \cline{2-4}
     & 3 & 17.83 & 3.66 &  &  &  &  &  &  &  &  \\ \cline{2-4}
     & 4 & 17.47 & 3.60 &  &  &  &  &  &  &  &  \\ \cline{2-4}
     & 5 & 17.60 & 3.93 &  &  &  &  &  &  &  &  \\ \hline
    \multirow{5}{*}{5} & 1 & 17.57 & 2.77 & \multirow{5}{*}{17.16} & \multirow{5}{*}{2.144} & \multirow{5}{*}{0.06} & \multirow{5}{*}{557} & \multirow{5}{*}{0.53} & \multirow{5}{*}{6.86} & \multirow{5}{*}{9.00} & \multirow{5}{*}{9.94} \\ \cline{2-4}
     & 2 & 16.87 & 2.09 &  &  &  &  &  &  &  &  \\ \cline{2-4}
     & 3 & 17.08 & 2.08 &  &  &  &  &  &  &  &  \\ \cline{2-4}
     & 4 & 16.33 & 2.08 &  &  &  &  &  &  &  &  \\ \cline{2-4}
     & 5 & 17.95 & 1.70 &  &  &  &  &  &  &  &  \\ \hline
    \multirow{5}{*}{6} & 1 & 13.59 & 2.19 & \multirow{5}{*}{13.71} & \multirow{5}{*}{2.142} & \multirow{5}{*}{0.07} & \multirow{5}{*}{655} & \multirow{5}{*}{0.66} & \multirow{5}{*}{7.72} & \multirow{5}{*}{7.40} & \multirow{5}{*}{9.90} \\ \cline{2-4}
     & 2 & 13.89 & 2.04 &  &  &  &  &  &  &  &  \\ \cline{2-4}
     & 3 & 14.20 & 2.15 &  &  &  &  &  &  &  &  \\ \cline{2-4}
     & 4 & 13.28 & 2.20 &  &  &  &  &  &  &  &  \\ \cline{2-4}
     & 5 & 13.59 & 2.13 &  &  &  &  &  &  &  &  \\ \hline
    \multirow{5}{*}{7} & 1 & 8.31 & 6.43 & \multirow{5}{*}{8.09} & \multirow{5}{*}{6.298} & \multirow{5}{*}{0.12} & \multirow{5}{*}{490} & \multirow{5}{*}{1.12} & \multirow{5}{*}{10.17} & \multirow{5}{*}{2.28} & \multirow{5}{*}{9.33} \\ \cline{2-4}
     & 2 & 8.15 & 6.18 &  &  &  &  &  &  &  &  \\ \cline{2-4}
     & 3 & 7.97 & 6.21 &  &  &  &  &  &  &  &  \\ \cline{2-4}
     & 4 & 7.86 & 6.42 &  &  &  &  &  &  &  &  \\ \cline{2-4}
     & 5 & 8.16 & 6.25 &  &  &  &  &  &  &  &  \\ \hline
    \multirow{5}{*}{8} & 1 & 15.85 & 13.84 & \multirow{5}{*}{15.554} & \multirow{5}{*}{13.98} & \multirow{5}{*}{0.06} & \multirow{5}{*}{691} & \multirow{5}{*}{0.58} & \multirow{5}{*}{7.23} & \multirow{5}{*}{2.11} & \multirow{5}{*}{2.20} \\ \cline{2-4}
     & 2 & 15.60 & 14.83 &  &  &  &  &  &  &  &  \\ \cline{2-4}
     & 3 & 15.37 & 13.59 &  &  &  &  &  &  &  &  \\ \cline{2-4}
     & 4 & 15.40 & 13.68 &  &  &  &  &  &  &  &  \\ \cline{2-4}
     & 5 & 15.55 & 13.96 &  &  &  &  &  &  &  &  \\ \hline
    \end{tabular}%
    }
    \caption{Observational data and calculation of $ne$ for the dynamic method}
    \label{obs1}    
\end{table*}
% Please add the following required packages to your document preamble:
% \usepackage{multirow}
% \usepackage{graphicx}
\begin{table*}[]
    \centering
    % \resizebox{\textwidth}{!}{%
    \begin{tabular}{|c|c|c|c|c|c|c|c|c|}
    \hline
    Drop no. & Sl no. & Free Fall& Mean Free fall  & Mean free fall velocity & Balancing & $\eta$  & r & $ne$ \\
     & & time (s) & time $t_f$ (s) (s) & ($v_f$) (in mm/s) & Voltage (V) & ($\times   10^{-13}$) & ($\times   10^{-7}$) (in m) & ($\times   10^{-19}$ C) \\ \hline
     \multirow{5}{*}{1} & 1 & 6.72 & \multirow{5}{*}{6.82} & \multirow{5}{*}{0.15} & \multirow{5}{*}{484} & \multirow{5}{*}{13.26} & \multirow{5}{*}{11.12} & \multirow{5}{*}{5.40} \\ \cline{2-3}
     & 2 & 6.97 &  &  &  &  &  &  \\ \cline{2-3}
     & 3 & 6.69 &  &  &  &  &  &  \\ \cline{2-3}
     & 4 & 6.81 &  &  &  &  &  &  \\ \cline{2-3}
     & 5 & 6.90 &  &  &  &  &  &  \\ \hline
    \multirow{5}{*}{2} & 1 & 8.78 & \multirow{5}{*}{8.32} & \multirow{5}{*}{0.12} & \multirow{5}{*}{208} & \multirow{5}{*}{10.87} & \multirow{5}{*}{10.03} & \multirow{5}{*}{9.21} \\ \cline{2-3}
     & 2 & 8.47 &  &  &  &  &  &  \\ \cline{2-3}
     & 3 & 8.03 &  &  &  &  &  &  \\ \cline{2-3}
     & 4 & 8.22 &  &  &  &  &  &  \\ \cline{2-3}
     & 5 & 8.09 &  &  &  &  &  &  \\ \hline
    \multirow{5}{*}{3} & 1 & 8.09 & \multirow{5}{*}{8.05} & \multirow{5}{*}{0.12} & \multirow{5}{*}{360} & \multirow{5}{*}{11.24} & \multirow{5}{*}{10.20} & \multirow{5}{*}{5.61} \\ \cline{2-3}
     & 2 & 8.03 &  &  &  &  &  &  \\ \cline{2-3}
     & 3 & 8.09 &  &  &  &  &  &  \\ \cline{2-3}
     & 4 & 7.97 &  &  &  &  &  &  \\ \cline{2-3}
     & 5 & 8.05 &  &  &  &  &  &  \\ \hline
    \multirow{5}{*}{4} & 1 & 6.34 & \multirow{5}{*}{6.22} & \multirow{5}{*}{0.16} & \multirow{5}{*}{523} & \multirow{5}{*}{14.53} & \multirow{5}{*}{11.65} & \multirow{5}{*}{5.75} \\ \cline{2-3}
     & 2 & 6.22 &  &  &  &  &  &  \\ \cline{2-3}
     & 3 & 6.22 &  &  &  &  &  &  \\ \cline{2-3}
     & 4 & 6.12 &  &  &  &  &  &  \\ \cline{2-3}
     & 5 & 6.21 &  &  &  &  &  &  \\ \hline
    \multirow{5}{*}{5} & 1 & 1.45 & \multirow{5}{*}{1.46} & \multirow{5}{*}{0.68} & \multirow{5}{*}{562} & \multirow{5}{*}{61.75} & \multirow{5}{*}{24.45} & \multirow{5}{*}{49.43} \\ \cline{2-3}
     & 2 & 1.48 &  &  &  &  &  &  \\ \cline{2-3}
     & 3 & 1.42 &  &  &  &  &  &  \\ \cline{2-3}
     & 4 & 1.50 &  &  &  &  &  &  \\ \cline{2-3}
     & 5 & 1.47 &  &  &  &  &  &  \\ \hline
    \multirow{5}{*}{6} & 1 & 15.35 & \multirow{5}{*}{14.37} & \multirow{5}{*}{0.07} & \multirow{5}{*}{146} & \multirow{5}{*}{6.29} & \multirow{5}{*}{7.54} & \multirow{5}{*}{5.58} \\ \cline{2-3}
     & 2 & 14.22 &  &  &  &  &  &  \\ \cline{2-3}
     & 3 & 13.73 &  &  &  &  &  &  \\ \cline{2-3}
     & 4 & 14.46 &  &  &  &  &  &  \\ \cline{2-3}
     & 5 & 14.07 &  &  &  &  &  &  \\ \hline
    \multirow{5}{*}{8} & 1 & 15.85 & \multirow{5}{*}{15.55} & \multirow{5}{*}{0.06} & \multirow{5}{*}{317} & \multirow{5}{*}{5.81} & \multirow{5}{*}{7.23} & \multirow{5}{*}{2.27} \\ \cline{2-3}
     & 2 & 15.60 &  &  &  &  &  &  \\ \cline{2-3}
     & 3 & 15.37 &  &  &  &  &  &  \\ \cline{2-3}
     & 4 & 15.40 &  &  &  &  &  &  \\ \cline{2-3}
     & 5 & 15.55 &  &  &  &  &  &  \\ \hline
    \multirow{5}{*}{7} & 1 & 8.31 & \multirow{5}{*}{8.07} & \multirow{5}{*}{0.12} & \multirow{5}{*}{214} & \multirow{5}{*}{11.20} & \multirow{5}{*}{10.18} & \multirow{5}{*}{9.38} \\ \cline{2-3}
     & 2 & 8.15 &  &  &  &  &  &  \\ \cline{2-3}
     & 3 & 7.86 &  &  &  &  &  &  \\ \cline{2-3}
     & 4 & 7.89 &  &  &  &  &  &  \\ \cline{2-3}
     & 5 & 8.16 &  &  &  &  &  &  \\ \hline
    \end{tabular}%
    % }
    \caption{Observational data and calculation of $ne$ for the balancing method}
    \label{obs2}    
\end{table*}

For tables III and IV, $ne$ divided was calculated by dividing the value of the charge $ne$ on all the droplets by minimum value, which was then rounded off to the nearest integer as
$n_\text{eff}$. This is the number of electrons present on a particular droplet. Now, by dividing $ne$ by $n_\text{eff}$ for each droplet, we can measure the value of $e$.

\begin{table}[H]
    \centering
    \begin{tabular}{|ccc|c|}
    \hline
    \multicolumn{1}{|c|}{ne} & \multicolumn{1}{c|}{$ne$} & \multirow{2}{*}{$n_\text{eff}$} & $ne/n_\text{eff}$ \\ 
    \multicolumn{1}{|c|}{($\times   10^{-19}$ C)} & \multicolumn{1}{c|}{divided} &  & ($\times 10^{-19}$ C) \\ \hline
    \multicolumn{1}{|c|}{51.51} & \multicolumn{1}{c|}{23.46} & 23 & 2.24 \\ \hline
    \multicolumn{1}{|c|}{60.47} & \multicolumn{1}{c|}{27.54} & 28 & 2.16 \\ \hline
    \multicolumn{1}{|c|}{8.33} & \multicolumn{1}{c|}{3.79} & 4 & 2.08 \\ \hline
    \multicolumn{1}{|c|}{8.12} & \multicolumn{1}{c|}{3.70} & 4 & 2.03 \\ \hline
    \multicolumn{1}{|c|}{9.94} & \multicolumn{1}{c|}{4.53} & 5 & 1.99 \\ \hline
    \multicolumn{1}{|c|}{9.90} & \multicolumn{1}{c|}{4.51} & 5 & 1.98 \\ \hline
    \multicolumn{1}{|c|}{9.33} & \multicolumn{1}{c|}{4.25} & 4 & 2.33 \\ \hline
    \multicolumn{1}{|c|}{2.20} & \multicolumn{1}{c|}{1.00} & 1 & 2.20 \\ \hline
    \multicolumn{3}{|c|}{Average $ne/n_\text{eff}$} & 2.13 \\ \hline
    \end{tabular}
    \caption{Calculation of $e$ for from dynamic method data}
\end{table}

\begin{table}[H]
    \centering
    \begin{tabular}{|ccc|c|}
    \hline
    \multicolumn{1}{|c|}{ne} & \multicolumn{1}{c|}{$ne$} & \multirow{2}{*}{$n_\text{eff}$} & $ne/n_\text{eff}$ \\ 
    \multicolumn{1}{|c|}{($\times   10^{-19}$ C)} & \multicolumn{1}{c|}{divided} &  & ($\times 10^{-19}$ C) \\ \hline
    \multicolumn{1}{|c|}{5.40} & \multicolumn{1}{c|}{2.38} & 2 & 2.70 \\ \hline
    \multicolumn{1}{|c|}{9.21} & \multicolumn{1}{c|}{4.07} & 4 & 2.30 \\ \hline
    \multicolumn{1}{|c|}{5.61} & \multicolumn{1}{c|}{2.48} & 2 & 2.80 \\ \hline
    \multicolumn{1}{|c|}{5.75} & \multicolumn{1}{c|}{2.54} & 3 & 1.92 \\ \hline
    \multicolumn{1}{|c|}{49.43} & \multicolumn{1}{c|}{21.82} & 22 & 2.25 \\ \hline
    \multicolumn{1}{|c|}{5.58} & \multicolumn{1}{c|}{2.46} & 2 & 2.79 \\ \hline
    \multicolumn{1}{|c|}{2.27} & \multicolumn{1}{c|}{1.00} & 1 & 2.27 \\ \hline
    \multicolumn{1}{|c|}{9.38} & \multicolumn{1}{c|}{4.14} & 4 & 2.35 \\ \hline
    \multicolumn{3}{|c|}{Average $ne/n_\text{eff}$} & 2.42 \\ \hline
    \end{tabular}
    \caption{Calculation of $e$ for from balancing method data}
\end{table}


\section{Error Analysis \&\\ Sources of Error}

We can calculate the error in our measurement of $e$ using the standard deviation method as follows,

\begin{align}
    \sigma_y = \sqrt{\frac{\sum (y_i - y_{avg})^2}{N-2}}
\end{align}

Using this for tables III and IV, the uncertainities in $e$ are,
\begin{itemize}
    \item For dynamic method, $\sigma_e=0.13\times 10^{-19}$ C
    \item For balancing method, $\sigma_e=0.31\times 10^{-19}$ C
\end{itemize}

As we can see from our error analysis that our experiment
was precise but not accurate. Measurements were consistent but systematically deviated from the true value
of the elementary charge ($e$). This suggests the presence
of systematic errors rather more than random errors. Possible
sources of such errors include:\\

\begin{itemize}
    \item The experiment relies on an accurate value of the viscosity of air to calculate the charge on the droplets.
    If the air temperature or humidity was not controlled
    or the viscosity value was incorrect, this could lead
    to systematic errors.
    \item Errors in determining the radius of the droplets, perhaps due to optical issues or incorrect assumptions
    about the droplet’s shape, could result in incorrect
    calculations of the charge.
    \item Changes in atmospheric pressure could affect the
    buoyant force on the droplets, leading to errors in
    the calculation of the charge.
    \item If the voltage supply or the distance between the
    plates is not accurately calibrated, the calculated
    charge values will be systematically off, leading to
    consistent but inaccurate results.
    \item If the oil droplets were contaminated with dust or
    other particles, this could alter their mass and the
    calculated charge, leading to consistently skewed results.
    \item An inaccurate voltmeter or fluctuating voltage supply could lead to an incorrect calculation of the electric field, affecting the determination of the charge
    on the droplets.
\end{itemize}