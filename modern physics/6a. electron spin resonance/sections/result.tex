\section{Discussion and Conclusion}

The static magnetic field was measured as,
\begin{align*}
    H_0 = (4.07 \pm 0.13) \text{ Gauss}
\end{align*}
And the Lande's factor was measured as, 
\begin{align*}
    g = (2.46 \pm 0.08)
\end{align*}
This is quite close to the literature value of $g= 2.0$. The error is due to the precision in the measurement of Q and P. Very small changes in Q at very high current is also not properly resolved leading to errors. Also, the phase matching of the four peaks is done by eye approximation. Errors may also come from the incorrect phase matching.

\section{Precautions}

    \begin{enumerate}
        \item Set up the experiment
        in a quiet location free from mechanical and
        electrical disturbances.
        \item Adjust the sensitivities of the X and Y plates of the oscilloscope so that the measurements remain within
        the linear range.
        \item Use a
        shielded cable to connect the Y-output from the
        ESR Spectrometer to minimize external noise
        and interference.
        \item Do not allow high currents (around 200
        mA) to flow through the Helmholtz coils for
        a prolonged time. This can cause unnecessary
        heating of the coils, potentially damaging them.
        \item Do not use an AC stabilizer, as it may distort the sinusoidal waveform. If necessary, use a variac to regulate the
        voltage.
        \item If the
        peaks do not align on the x-scale, check the sinusoidal waveform of the mains voltage. Distortion may occur due to overloading of the main
        line by other heavy devices operating on the
        same circuit.
    \end{enumerate}

