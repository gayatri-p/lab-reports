\section{Procedure}

\subsection*{Determination of $g$}

\begin{enumerate}
    \item Adjust the spectrometer and use a spirit level to make it parallel. Fix the grating on the prism table.
    \item Bring in the Hg source close to the collimator of the and switch ON the power supply and let the Hg lamp warm up.
    \item Look through the telescope to notice the first order spectral lines of Hg on both sides of the direct image of the slit at the center.
    Make the spectral lines vertical by turning the grating slightly in its plane.
    \item Note down the positions of the cross wire for each line on one side using the two
    verniers on the spectrometer.
    \item Repeat the above step by turning the telescope to the other side too. Determine the
    diffraction angle, $\theta$, for all the spectral lines of Hg spectrum. Using the
    spectral data of Hg provided, calculate $g$.
    \item Switch off the power supply and remove the Hg spectral tube.
\end{enumerate}

\subsection*{Studying hydrogen spectrum}

\begin{enumerate}
    \item Bring in the hydrogen source to the front of collimator. Then switch ON the
    power supply.
    \item Look through the telescope to notice the three first order spectral lines of Hydrogen Balmer series (red, green and violet) on both sides of the direct image of the slit at the center.
    \item Follow the step described in the above section to determine the diffraction angle for all three lines.
    \item Using the value of g determined earlier, calculate the wavelength of each of the
    spectral lines.
    \item Switch off hydrogen spectral tube.
\end{enumerate}