\begin{table}[H]
    \centering
    \begin{tabular}{cccc}\hline
        \multicolumn{4}{c}{Radii ($\mu$m)}             \\\hline
        & $R_1$& $R_2$ & $R_3$  \\
        & (1st order)&(2nd order) & (3rd order) \\ \hline        
        \multicolumn{4}{l}{Pole separation = 40 mm}             \\\hline
        a &  30.54         &         139.47  &170.47           \\
        b &    72.16       &      140.53     &      180.27     \\
        c &       98.66    &159.69           &         188.19   \\\hline
        \multicolumn{4}{l}{Pole separation = 41 mm}             \\\hline
        a &    38.30       &128.51           &           168.97\\
        b &        71.16   &      139.47     &      180.44     \\
        c &      98.66     &           150.08&183.59            \\\hline
        \multicolumn{4}{l}{Pole separation = 42 mm}             \\ \hline
        a &   28.37        &   126.90        &168.48           \\
        b &    63.49       &    135.73        &     175.78      \\
        c &   85.02        &145.07           &182.98            \\\hline
        \multicolumn{4}{l}{Pole separation = 44 mm}             \\\hline
        a &  38.98         &          128.66 &168.82           \\
        b &      63.85     &      136.56     &      173.26     \\
        c &        82.02   & 143.55          &           178.39 \\\hline
        \multicolumn{4}{l}{Pole separation = 45 mm}             \\\hline
        a &  47.91         &         132.32  &173.78           \\
        b &      67.57     &    139.61       &     177.49      \\
        c &          82.58 &144.96           &          182.39  \\\hline
        \end{tabular}    
        \caption{Radii measured for each order for different values of pole separation. Here, Components of each order rings are designated as $a$, $b$ and $c$, where the outer $a$ and $c$ are $\sigma$ lines and the middle $b$ is $\pi$ line}
        \label{tab:1}
    \end{table}