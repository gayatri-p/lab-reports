\section{Procedure}

\subsection{Normal Zeeman Effect\\w/ Transverse Magnetic Field}

\begin{enumerate}
    \item Switch on the Cd lamp and wait for it to warm up.
    \item Align the lamp, first 50 mm lens and Fabry-perot tube. Use a red filter so that we only see the splitting of the 643.8 nm line. Observe the ring pattern coming out from etalon with your naked eye.
    \item Now place the 300 mm lens and 50 mm lens along with camera and align the entire setup. Adjust the camera settings to make sure the rings of first three orders in normal Zeeman effect are visible on the screen.
    \item Adjust the CMOS camera settings to get the ring pattern at the
    center of the screen. 
    \item One can observe that the splitting is function of magnetic field by varying the distance between pole pieces. 
    \item After confirming the ring pattern of Normal Zeeman effect with the ring pattern
    up to 3 orders, take pictures using the software for different values of magnetic field.
    \item Now, place the polarizer and notice that middle line of each order is horizontally
    polarized and outer rings are vertically polarized. Identify the $\sigma$ and $\pi$ lines. 
    %  Hence, one can identify the outer ones as $\sigma$ lines and the middle one as a $\pi$ line. 
\end{enumerate}

\subsection{Normal Zeeman Effect\\w/ Longitudinal Magnetic Field}

\begin{enumerate}
    \item Now, rotate the magnetic field such that it is in the longitudinal direction and realign the optical components.
    \item Place the quarter wave place and align it. This can convert linearly polarized light to circularly polarized light and vice-versa.
    \item By rotating the polarizer, one can identify the left and right circularly polarized light.
\end{enumerate}

\subsection{Anomalous Zeeman Effect\\w/ Transverse Magnetic Field}

\begin{enumerate}
    \item Now, rotate the magnetic field back to the transverse direction. Replace the red filter with a green filter such that we are observing the 508.58 nm line.
    \item  Align the optics and observe 8 rings for each order (due to the insufficient resolution of Fabry perot etalon, instead of 9 rings, we see only 8 or sometimes 7 rings.)
    \item Identify the $\sigma$ and $\pi$ lines and record pictures.
\end{enumerate}

\subsection{Anomalous Zeeman Effect\\w/ Longitudinal Magnetic Field}

\begin{enumerate}
    \item Now, rotate the magnetic field along the longitudinal direction and realign the optical components.
    \item Place the quarter wave place and align it as well.
    \item By rotating the polarizer, identify the left and right circularly polarized $\sigma$ lines.
\end{enumerate}