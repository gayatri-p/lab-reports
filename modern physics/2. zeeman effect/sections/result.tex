\section{Result and Discussion}
Using a a Cd lamp source and a Fabry Perot etalon, we have observed Normal and Anomalous Zeeman effect in action. By observing upto 3 orders of rings for normal Zeeman effect with magnetic field in transverse direction (Fig. \ref{f1}), we have calculated the value of Bohr's Magneton $\mu_B$ as,

\begin{align*}
    \mu_B = (5.215 \pm 0.293) \times 10^{-24} \text{ J/T}
\end{align*}

The value of $\mu_B$ obtained however is quite away from the actual value. The primary reason could be beacuse these magnets are not properly calibrated with the data graph
provided to us. Maybe these magnets are old and
have degraded over time. 

Furthermore on observing the rings through a polarizing filter, we can observe that the middle line of each order is horizontally
polarized and outer rings are vertically polarized. Identify the $\sigma$ and $\pi$ lines. 
Hence, one can identify the outer ones as $\sigma$ lines and the middle one as a $\pi$ line.

In the case of normal longitudinal Zeeman Effect, we first convert the circularly polarized light into linearly polarization using a quarter-wave plate. We observe that only the $\sigma$ lines are now visible. Now, by using the polarizing filter, one can identify the left and right circularly polarized light as $\sigma^+$ and $\sigma^-$ lines (Fig. \ref{f2}). The reason we don't see the $\pi$ line is beacause in this case the light propagates along $B$, and the incident electromagnetic wave has no oscillatory component along the linear
oscillator parallel to $B$ (the $\pi$ component), so absorption of that component cannot occur (Fig. \ref{fields}).

In the case of anomalous transverse Zeeman effect, we were able to identify 7 to 8 lines per order (Fig. \ref{f3}). After using the polarization filter, we could distinguish between the 5 $\sigma$ lines and 3 $\pi$ lines by rotating the polarizer. For the anomalous transverse Zeeman effect, we could only observe the 5 $\sigma$ lines (Fig. \ref{ano_long}). The reason for it is the same as that for normal Zeeman effect.

\section{Precautions and Sources of Error}

    \begin{enumerate}
        \item Handle the filters carefully.
        \item While setting the optical path don’t look to the Cd-lamp in naked eye for long.
        \item Make sure the optical components are properly aligned and the pattern in is the center before taking measurements.
    \end{enumerate}

\section{Conclusion}
By subjecting a Cadmium lamp to various magnetic
field intensities we have studied normal and anomalous Zeeman effect, with the help of a Fabry-Perot etalon. 
% The amount of splitting is measured by the physical distance between the spectral lines. 
We have also studied the various components of Zeeman lines and their respective polarizations by varying the direction of the applied magnetic field.
We have estimated the value of Bohr's magneton from the results.