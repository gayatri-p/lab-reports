\section{Discussion and Conclusion}
The relation between magnetic field and the resonance frequency followed the expected trend. The fitted line is given by $f_\text{res} = (4.77 \pm 0.07)\sqrt{B_{PM}}$.

Additionally, knowing the fitted parameter and the moment of inertia of the compass needle will allow us to determine the magnetic moment of the compass or vice versa. This experiment qualitatively explored the relationship between resonance frequency and external magnetic field. Magnetic resonance is a phenomenon involving the resonant excitation of particle spins, which can be those of atomic nuclei or electrons. These microscopic particles possess intrinsic spin, resulting in magnetic moments. In a magnetic field, these moments experience a torque that aligns them with the field. An additional magnetic drive induces oscillations in these moments, leading to magnetic resonance.

\section{Precautions}

    \begin{enumerate}
        \item After finding the approximate distance move the magnet to and fro to get the precise location.
        \item Change the position of magnet slowly in order.
        to get the proper resonance condition.
        \item  It must be ensure that magnet and coil are properly aligned with respect to each other.
        \item At each position the needle must be given enough time to reach a steady state of oscillation.
    \end{enumerate}

