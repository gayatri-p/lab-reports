\section{Procedure}

\subsection*{Calibration of the spectrometer}

\begin{enumerate}
    \item Level the constant deviation spectrometer by means of a sprit level and focus the
    telescope.
    \item Place the constant deviation prism on the prism table so that its 90$^{\circ}$ vertex faces
    towards the objective of the telescope.
    \item Now, the Hg lamp is placed in front of the collimator
    \item The drum is rotated so that it reads a value closest to the value of the single and
    prominent green line. Now, by slightly rotating the prism and looking through the
    telescope, the green line is made to coincide on the pointer
    \item Once the spectral line coincides with the pointer, clamp the prism. Now, note down
    the closest values read by the drum ($\lambda_\text{obs}$) corresponding to the given values
    ($\lambda_\text{given}$) of different wavelength of mercury. Plot $\lambda_\text{obs} \sim \lambda_\text{given}$ and fit it with a
    straight line to obtain the calibration parameters. Using these parameters any observed
    unknown wavelength can be calculated to determine the correct wavelength $\lambda_\text{corr}$.
\end{enumerate}

\subsection*{Studying the Emission Spectra of Metals}

\begin{enumerate}
    \item Replace the calibration source by the arc source using a particular metal arc of
    interest.
    \item The D.C.power supply is connected to the arc stand holding the pointed metal arc one
    over the other. Switch ON the power supply and observe the arc begins to glow
    \item The spectrum is observed in the CDS. Adjust the drum head to make the
    pointer/crosswire coincide on each of the spectral lines and read the characteristic
    wavelength of the different lines emitted by the metals directly.
    \item Compare the values of spectral lines obtained for different metals with the literature
    values.
\end{enumerate}

\subsection*{Studying the Absorption Spectra of Iodine Vapour}

\begin{enumerate}
    \item Place an incandescent lamp
    as the source and observe the continuous spectrum
    \item The D.C.power supply is connected to the arc stand holding the pointed metal arc one
    over the other. Switch ON the power supply and observe the arc begins to glow
    \item The spectrum is observed in the CDS. Adjust the drum head to make the
    pointer/crosswire coincide on each of the spectral lines and read the characteristic
    wavelength of the different lines emitted by the metals directly.
    \item Compare the values of spectral lines obtained for different metals with the literature
    values.
\end{enumerate}