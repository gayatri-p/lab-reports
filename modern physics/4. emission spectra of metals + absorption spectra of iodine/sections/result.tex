\section{Discussion and Conclusion}

For the emission spectra of metals, comparing the literature value of the wavelengths with the observed wavelength, we find that they are quite close. For Copper, we found almost all the major wavelengths mentioned in the literature. Since Brass is a mixture of copper and zinc, we observed both the characteristic wavelengths of copper and zinc in brass. This tells us, that, in Brass, the chemical properties of the two metals remains the same.

From the absorption spectrum of Iodine, we calculated the following parameters,

\begin{itemize}
    \item Bond dissociation energy
    \begin{align*}
        D_0 = (0.240 \pm 0.002) \text{ eV/molecule}
    \end{align*}
    \item Force constant
    \begin{align*}
        f=(40.30 \pm 19.35) \text{ N m}^{-1}
    \end{align*}
\end{itemize}

Here we can see that the value of $f$ is close to the literature value of $41\; N m^{-1}$ while $D_0$ is far from the literature value of $1.54$ eV/molecule. The error in bond dissociation energy can be associated with the fact that during the experiment the line weren't sharply visible and the intensity continued to decrease, thereby lowering the range of observation and thus the difference of energy between the highest and lowest energy state.

The error in the calculation of $f$ is due to the large value of least count of the device. Also, there must have been a lot of random errors involved. The source of one such error might be due to the fact that the fringes were not perfectly focussed, and while taking a reading, we might have taken reading at a position slightly beside the position of the actual minima.

\section{Precautions}

    \begin{enumerate}
        \item Handle the metal arcs carefully
        \item Turn the drum only in one direction to avoid backlash error
        \item Make sure the telescope is properly focused and that the crosswire is parallel, before taking any readings.
    \end{enumerate}
