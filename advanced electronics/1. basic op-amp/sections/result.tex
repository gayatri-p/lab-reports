\section{Discussion and Sources of Error}
Using an IC 741 op-amp, we have successfully constructed inverting and non-inverting amplifier setups and measured their gain values. In case of an inverting amplifier, there is about 0.2\% and 0.4\% deviation of the experimental value from the theoretical value for each set of $R_f$ and $R_\text{in}$ respectively. For the non-inverting setup, the deviations were 0.4\% and 1.7\% respectively. These deviations could be due ito error in the measurement of resistance, error stemming from assumption of an ideal op-amp or due to small capacitances builing up at the metal junctions in the circuit.

Additionally, we have also demonstrated op-amp as a summing and difference amplifier. While the difference in theoretical and experimental values in the case of the summing amplifier was at most $\pm$0.06 V, the same for the difference amplifier was 0.035V to 0.194V. Since, there is no ideal op-amp in real life, the deviations could be due to a multitude of factors including common-mode gain, offset voltage, noise or due to supply voltage fluctuations. 

We have also used the op-amp to build a comparator circuit with and without threshold voltage as demonstrated in Fig. \ref{g3}. Similarly we have built a schmitt trigger circuit which was able to show hysteresis as shown in Fig. \ref{g4} and Fig. \ref{g5}. 

\section{Precautions}

\begin{enumerate}
    \item The power supply to the operational amplifier never becomes reversed in polarity.
    \item Make sure the connections are proper before switching on the circuit.
\end{enumerate}

\section{Conclusion}
We have demonstrated basic op-amp configurations, including inverting and non-inverting amplifiers, and its application as a summing amplifier, a difference amplifier, a comparator and a Schmitt trigger.

\section{Applications}
Op-amps have a broad range of usages, and as such are a key building block in many analog applications — including filter designs, voltage buffers, comparator circuits etc. 

Early op-amps were used in circuits that could add, subtract, multiply, and even solve differential equations.  General purpose op amps were redesigned to optimize or add certain features, and now can be found in communication ICs, radio ICs, audio/video ICs, instrumentation circuits which require high sensitivity etc.