\section{Discussion \& Conclusion}
In this experiment, we have successfully built and studied the properties of multivibrator circuits using the IC 555 timer.

We built an astable multivibrator where the astable function was achieved through acharging and discharging capacitors through resistors and we varied the duty cycle and the frequency of the resulting rectangular wave by varying the resistance and capacitance values.
Using a potentiometer, we were able to vary the duty cycle of the output wave from 6.3\% to 79.3\% around an oscillating frequency of 2.574 kHz.

We also built a monostable multivibrator with a pulse width of 10.96s, which was triggered by a negative pulse applied to the \verb|TRIGGER| pin. Furthermore, we also built a bistable multivibrator with two stable states, \verb|SET| and \verb|RESET|, both of which are again externally triggered.

\section{Precautions}

\begin{enumerate}
    \item Make sure all the connections are proper before switching on the circuit.
    \item Make sure to give 10V as the power supply ($V_{CC}$).
    \item Make sure to ground the oscilloscope probes or else they will show distorted output.
\end{enumerate}

\section{Applications}
Astable multivibrators have widespread usage across a variety of domains like
communication systems -- for frequency generation and modulation, 
in timing and control circuits, in
testing equipment -- to serve as signal generators to provide inputs for device testing, and in power supply systems -- where they act as oscillators.
