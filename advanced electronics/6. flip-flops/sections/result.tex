\section{Discussion \& Conclusion}
In this experiment we contructed and studied sequential logic circuits using various kinds of flip-flop circuits. We discussed simple RS flip-flop using both NOR and NAND gates, and its practical application in switch debouncing circuits. We discussed RS flip-flops with a clock signal as well as D flip-flops which are able to remove the indeterminate state in RS flip-flops.

Furthermore, we dicussed JK flip-flops and its purpose, including toggle action and how to counteract the racing problem using a master-slave flip-flop. We also discussed a modified version of the master-slave JK flip-flop, the T flip-flop, which only performs \verb|TOGGLE| and \verb|HOLD| actions.

\section{Precautions}

\begin{enumerate}
    \item Make sure the connections are proper before switching on the circuit.
    \item Make sure to connect a resistor parallel to the LED as to not burn out the LED.
\end{enumerate}

\section{Applications}
Flip-flops play a critical role in computer electronics by serving as memory elements, storing state information, ensuring clock synchronization, enabling digital counting, and facilitating control logic. They are essential for data storage, sequencing, coordination, and control within a computer system.