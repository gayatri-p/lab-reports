\section{Observations}

The ICs were powered by 5V DC, \verb|HIGH| inputs from the 5V line, \verb|LOW| inputs from ground. Outputs were monitored with LEDs (LED on means 1, 0 otherwise).

\subsection{Half Adder Circuit}

\noindent Using the circuit given in Fig. \ref{half-add}.
\begin{table}[H]
    \centering
    \begin{tabular}{|c|c|c|c|}\hline
    A & B & Q & B$_N$ \\ \hline
    0 & 0 & 0 & 0 \\ 
    0 & 1 & 1 & 1 \\ 
    1 & 0 & 1 & 0 \\ 
    1 & 1 & 0 & 0 \\ \hline
    \end{tabular}
    \caption{Observation table for the half adder circuit}
\end{table}

\subsection{Full Adder Circuit}

\noindent Using the circuit given in Fig. \ref{full-add}.
\begin{table}[H]
    \centering
    \begin{tabular}{|c|c|c|c|c|}\hline
        B$_{N-1}$ & A & B & Q & B$_N$ \\ \hline
        0 & 0 & 0 & 0 & 0 \\ 
        0 & 0 & 1 & 1 & 1 \\ 
        0 & 1 & 0 & 1 & 0 \\ 
        0 & 1 & 1 & 0 & 0 \\
        1 & 0 & 0 & 1 & 1 \\ 
        1 & 0 & 1 & 0 & 1 \\ 
        1 & 1 & 0 & 0 & 0 \\ 
        1 & 1 & 1 & 1 & 1 \\ \hline
    \end{tabular}
    \caption{Observation table for the full adder circuit}
\end{table}

\subsection{Half Subtractor Circuit}

\noindent Using the circuit given in Fig. \ref{half-subt}.
\begin{table}[H]
    \centering
    \begin{tabular}{|c|c|c|c|}\hline
        A & B & Q & B$_N$ \\ \hline
        0 & 0 & 0 & 0 \\ 
        0 & 1 & 1 & 1 \\ 
        1 & 0 & 1 & 0 \\ 
        1 & 1 & 0 & 0 \\ \hline
    \end{tabular}
    \caption{Observation table for the half subtractor circuit}
\end{table}

\subsection{Full Subtractor Circuit}

\noindent Using the circuit given in Fig. \ref{full-subt}.
\begin{table}[H]
    \centering
    \begin{tabular}{|c|c|c|c|c|}\hline
        B$_{N-1}$ & A & B & Q & B$_N$ \\ \hline
        0 & 0 & 0 & 0 & 0 \\ 
        0 & 0 & 1 & 1 & 1 \\ 
        0 & 1 & 0 & 1 & 0 \\ 
        0 & 1 & 1 & 0 & 0 \\
        1 & 0 & 0 & 1 & 1 \\ 
        1 & 0 & 1 & 0 & 1 \\ 
        1 & 1 & 0 & 0 & 0 \\ 
        1 & 1 & 1 & 1 & 1 \\ \hline
    \end{tabular}
    \caption{Observation table for the full subtractor circuit}
\end{table}

\newpage
\subsection{Full Adder-Subtractor Circuit}
\newcolumntype{Y}{>{\centering\arraybackslash}X}
\noindent Using the circuit given in Fig. \ref{full-add-sub}.
\begin{table}[H]
    \centering
    \begin{tabularx}{0.55\columnwidth}{|Y|Y|Y|Y|Y|}\hline
        % C$_{N-1}$ & A & B & Q & C$_N$ \\ \/hline
        B$_{N-1}$ & A & B & Q & B$_N$ \\ \hline
        \multicolumn{5}{|l|}{$X_1=X_2=1$ (Full Subtractor)}\\ \hline
        0 & 0 & 0 & 0 & 0 \\ 
        0 & 0 & 1 & 1 & 1 \\ 
        0 & 1 & 0 & 1 & 0 \\ 
        0 & 1 & 1 & 0 & 0 \\
        1 & 0 & 0 & 1 & 1 \\ 
        1 & 0 & 1 & 0 & 1 \\ 
        1 & 1 & 0 & 0 & 0 \\ 
        1 & 1 & 1 & 1 & 1 \\ \hline
        \multicolumn{5}{|l|}{$X_1=X_2=0$ (Full Adder)}\\ \hline
        0 & 0 & 0 & 0 & 0 \\ 
        0 & 0 & 1 & 1 & 1 \\ 
        0 & 1 & 0 & 1 & 0 \\ 
        0 & 1 & 1 & 0 & 0 \\
        1 & 0 & 0 & 1 & 1 \\ 
        1 & 0 & 1 & 0 & 1 \\ 
        1 & 1 & 0 & 0 & 0 \\ 
        1 & 1 & 1 & 1 & 1 \\ \hline
    \end{tabularx}
    \caption{Observation table for the full adder-subtractor circuit\\}
\end{table}