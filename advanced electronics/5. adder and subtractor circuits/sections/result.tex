\section{Discussion \& Conclusion}
% In this experiment, we learnt abour binary addition and subtraction operations and its practical implementation using different logic gates. We discussed the need for carry and borrow bits during and addition and subtraction respectfully as well as how to integrate the borrow bits of a previous operation using full adder/subtractor circuits.

% We also discuss how to build an N-bit adder/subtractor circuit using ripple carry circuits.

% Finally, we built a combined adder-subtractor circuit which can perform either addition or subtraction at once based on a control signal we provide.

In this experiment we explored binary addition and subtraction and their practical implementation using logic gates. We discussed construction of N-bit adders/subtractors using ripple carry circuits. We also discussed the need for carry and borrow bits, their integration in full adder/subtractor circuits, and finally designing a combined adder-subtractor controlled by a signal.

\section{Precautions}

\begin{enumerate}
    \item Make sure the connections are proper before switching on the circuit.
    \item Make sure to connect a resistor parallel to the LED as to not burn out the LED.
\end{enumerate}

\section{Applications}
Adders \& Subtractors are wildly used in a computer's ALU (Arithmetic and Logic Unit) to perform operations as well as in CPUs and GPUs.
They are also used in microcontrollers for arithmetic additions, program counters, timers and in networking and digital signal processing systems.