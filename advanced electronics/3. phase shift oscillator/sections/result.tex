\section{Discussion and Conclusion}

We have successfully constructed and demonstrated a phase shift oscillator using an op-amp and an RC network. We found out that over a minimum gain of around $|A_V|=37.45$, the oscillator started showing stable periodic sinusoidal oscillations.

The observed value of the cutoff frequency of the oscillator was found to be 719.4 Hz, which shows around 5.8\% deviation from the theoretical value. A major reason for this could be the error in the calculation since the average values of R and C were used in the calculation of cutoff frequency.

Also by using a secondary input source (function generator), were able to observe Lisssajous figures on the oscilloscope, at and near the cutoff frequency as well as near its higher harmonics.

We can thus build an oscillator circuit using op-amps which can be used to produce very low frequencies. It does not require transformers or inductors, and the circuit provides good frequency stability. These factors make them ideal for small circuits that require precision.

\section{Precautions \& Sources of Error}

\begin{enumerate}
    \item Make sure the connections are proper before switching on the circuit.
    \item Change the value of the potentiometer very carefully.
    \item Make sure that the R and C values in the RC network are close, to get accurate values of $f_c$.
\end{enumerate}

\section{Applications}
Oscillators are used by all kinds of laptop or smartphone processors to generate clock signals. Radio and mobile receivers use them to generate local carrier frequency signal.

RC phase shift oscillators in particular can be used for low-frequency applications, such as devices that produce radio and audio frequencies.